\chapter{Introduction}

Understanding human behaviour is one of the main goals of artificial intelligence research.
One approach towards achieving this goal is called Human Activity Recognition. 
Human Activity Recognition (HAR) is the process of recognizing specific gestures or actions performed by humans from different sources, e.g., images, videos or sensor data.
Knowledge of actions performed by humans is useful in different application contexts, such as video surveillance, human-machine interaction or
for evaluating worker performance in a warehouse setting \cite{reining_towards_2018}.

One approach for predicting human actions is to use pose information as an input to a HAR machine learning model, such as a neural network.
A pose is a set of keypoints on the person's body, usually joints between limbs.
For images, a pose is usually given by a set of pixel coordinates.
Each pixel coordinate refers to one estimated joint position.
Poses are estimated using a dedicated machine learning model, which is often based on neural networks as well.
In \cite{jhuang_towards_2013}, the authors find that using human pose information is useful when training HAR models, more so than other features such as image features.

In Human Activity Recognition research, a pose estimation model is often used to precompute the pose for each image.
In \cite{luvizon_2d/3d_2018}, the authors argue that jointly learning the pose and the action using a convolutional neural networks may improve the learning process of the action predictor.
This approach of jointly learning was previously not possible with many pose estimators since the output needed additional postprocessing steps.
Many pose estimation models output joint heatmaps, which contain the likelihood of a joint being present for each pixel in the input image.
A function called argmax extracts the pixel coordinates with the highest likelihood from the joint heatmaps.
The argmax function is, however, not differentiable.
The optimization algorithms used for training neural networks require a fully differentiable network for learning the parameters of the network.
With the introduction of the Soft-argmax function in \cite{luvizon_human_2017}, the authors presented an approach for extracting exact pose from the joint heatmaps using a differentiable function.
This function makes it possible to train the pose estimator and action recognition model jointly.
% Objective

The first objective of this thesis is to use the Convolutional neural network model proposed by \cite{luvizon_2d/3d_2018} to recreate their results.
This includes investigating the performance of the network using different hyperparameters.
Second, the model is evaluated using a more challenging benchmark to gain a better understanding of how well the model performs on different data.
Third, the network is trained in an end-to-end approach.
The authors in \cite{luvizon_2d/3d_2018} use a technique called pretraining for the pose estimator part of the network.
Pretraining means that the pose estimator part of the network is trained indepently of the rest of the network.
The pretrained parameters of the pose estimator are then transfered to the complete network and the network is finetuned.
A different approach for training the network is called end-to-end learning and refers to the method where the parameters of the network are initialized randomly.
From this random initialization, the entire network is trained jointly.
While pretraining certain parts of a network often leads to a faster convergence of the model, it also adds more complexity to the training process, because two training processes need to be optimized.

% Outline
This thesis is divided into six chapters, including this introduction.
Chapter \ref{sec:chapter2} explains the fundamentals of Human Activity Recognition in the context of video data, pose estimation using image data as well as artificial and convolutional neural networks.
Recent relevant work in the fields of HAR and pose estimation are discussed in detail in Chapter \ref{sec:chapter3}.
There, a focus is set on Human Activity Recognition methods using pose information.
Next, a detailed explanation of the methods of \cite{luvizon_2d/3d_2018} are presented in Section \ref{sec:chapter4}.
Chapter \ref{sec:chapter4} also discusses limitations to the approach by \cite{luvizon_2d/3d_2018} and further experiments conducted in this thesis.
In Chapter \ref{sec:chapter5}, the experiments, as well as the used datasets and metrics, are explained.
Moreover, the results of the proposed experiments are discussed.
Finally, a conclusion of the findings from the experiments are presented in Chapter \ref{sec:chapter6}.