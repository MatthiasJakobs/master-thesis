\documentclass[twoside]{article}

\usepackage{lipsum} % Package to generate dummy text throughout this template

\usepackage[sc]{mathpazo} % Use the Palatino font
\usepackage[T1]{fontenc} % Use 8-bit encoding that has 256 glyphs
\linespread{1.05} % Line spacing - Palatino needs more space between lines
\usepackage{microtype} % Slightly tweak font spacing for aesthetics

\usepackage[hmarginratio=1:1,top=32mm,columnsep=20pt]{geometry} % Document margins
\usepackage{multicol} % Used for the two-column layout of the document
\usepackage[hang, small,labelfont=bf,up,textfont=it,up]{caption} % Custom captions under/above floats in tables or figures
\usepackage{booktabs} % Horizontal rules in tables
\usepackage{float} % Required for tables and figures in the multi-column environment - they need to be placed in specific locations with the [H] (e.g. \begin{table}[H])
\usepackage{hyperref} % For hyperlinks in the PDF

\usepackage{lettrine} % The lettrine is the first enlarged letter at the beginning of the text
\usepackage{paralist} % Used for the compactitem environment which makes bullet points with less space between them

\usepackage{abstract} % Allows abstract customization
\renewcommand{\abstractnamefont}{\normalfont\bfseries} % Set the "Abstract" text to bold
\renewcommand{\abstracttextfont}{\normalfont\small\itshape} % Set the abstract itself to small italic text

\usepackage{titlesec} % Allows customization of titles
\renewcommand\thesection{\Roman{section}} % Roman numerals for the sections
\renewcommand\thesubsection{\Roman{subsection}} % Roman numerals for subsections
\titleformat{\section}[block]{\large\scshape\centering}{\thesection.}{1em}{} % Change the look of the section titles
\titleformat{\subsection}[block]{\large}{\thesubsection.}{1em}{} % Change the look of the section titles

\usepackage{fancyhdr} % Headers and footers
\pagestyle{fancy} % All pages have headers and footers
\fancyhead{} % Blank out the default header
\fancyfoot{} % Blank out the default footer
\fancyfoot[RO,LE]{\thepage} % Custom footer text

\usepackage{cite}
\usepackage{paralist, tabularx}

%----------------------------------------------------------------------------------------
%	TITLE SECTION
%----------------------------------------------------------------------------------------

\title{\vspace{-15mm}\fontsize{24pt}{10pt}\selectfont\textbf{Master thesis proposal}} % Article title

\author{
\large
\textsc{Matthias Jakobs}\\[2mm] % Your name
\normalsize \href{mailto:matthias.jakobs@tu-dortmund.de}{matthias.jakobs@tu-dortmund.de} % Your email address
\vspace{-5mm}
}
\date{}

%----------------------------------------------------------------------------------------

\begin{document}

\maketitle % Insert title

\thispagestyle{fancy} % All pages have headers and footers

%----------------------------------------------------------------------------------------
%	ABSTRACT
%----------------------------------------------------------------------------------------

%\begin{abstract}
\vspace{20px}

%\end{abstract}

%----------------------------------------------------------------------------------------
%	ARTICLE CONTENTS
%----------------------------------------------------------------------------------------

\begin{multicols}{2} % Two-column layout throughout the main article text

\section{Motivation}

Detecting human actions from video without any extra inputs. Example: Picking in a warehouse \cite{reining_towards_2018}. There, motion capturing using dedicated devices on a subjects body is used to get skeleton data. This is time consuming and cumbersome. Better would be to get skeleton directly from video.
Poses can be incorporated into action recognition for better results \cite{khalid_multi-modal_2018}. But: Need good pose data.

occlusion:
    vital because not every joint is seen all the time



%------------------------------------------------
\section{Recent Work}
Approaches focusing on single image, single person:
    Stacked Hourglass \cite{newell_stacked_2016}
    Convolutional Pose Machines \cite{wei_convolutional_2016}

Approaches focusing on single image, multiple persons:

Approaches focusing on video:

Two- and Three stream models for action recognition could be used for just pose estimation

%------------------------------------------------
\section{Method}

Using deep models (convolutional neural networks)

Incorporate spatial and temporal information (ja? erstmal paper zu video lesen)

Try to make robust against occlusion (welche datensätze bieten das?)


%------------------------------------------------
\section{Experiments}
\subsection{Dataset}

\textbf{single person, single image:} Leeds Sport Pose(LSP) \cite{johnson_clustered_2010}, LSP-Extended \cite{johnson_learning_2011}, MPII Human Pose (Single person) \cite{andriluka_2d_2014} Looking Into Person (kann wohl auch joints) \cite{gong_look_2017}

\textbf{multi person, single image} PoseTrack \cite{andriluka_posetrack:_2018} (also video)

\subsection{Metric}

\textbf{single person, single image:} Percentage of Correct Keypoints (PCK), PCKh, Percent of Detected Joints (\cite{toshev_deeppose:_2014})


%------------------------------------------------
%   REFERENCE LIST
%------------------------------------------------
\bibliographystyle{acm}
\bibliography{bibliography}
%----------------------------------------------------------------------------------------

\end{multicols}

\end{document}
